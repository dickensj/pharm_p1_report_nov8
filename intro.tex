\documentclass[handout]{beamer}
%
% Choose how your presentation looks.
%
% For more themes, color themes and font themes, see:
% http://deic.uab.es/~iblanes/beamer_gallery/index_by_theme.html
%
\mode<presentation>
{
  \usetheme{default}      % or try Darmstadt, Madrid, Warsaw, ...
  \usecolortheme{default} % or try albatross, beaver, crane, ...
  \usefonttheme{default}  % or try serif, structurebold, ...
  \setbeamertemplate{navigation symbols}{}
  \setbeamertemplate{caption}[numbered]
}
\usepackage{hyperref}
\usepackage{lmodern}
\usepackage[english]{babel}
\usepackage[utf8x]{inputenc}
\usepackage{multirow}
\usepackage{color}
\definecolor{dg}{rgb}{0.29, 0.33, 0.13}

\title{Statistical Analysis of Intestinal Events, Ibuprofen Absorption and Plasma Levels}
% \author{Gordon Amidon${}^{1}$, Joey Dickens${}^{2}$, Kerby Shedden${}^{2}$}
% \institute{Departments of Pharmaceutical Sciences (1) and Statistics (2)\\ University of Michigan}
\date{11/08/17}

\setbeamertemplate{sidebar right}{}
\setbeamertemplate{footline}{%
\hfill\usebeamertemplate***{navigation symbols}
\hspace{1cm}\insertframenumber{}/\inserttotalframenumber}

\begin{document}

\begin{frame}
  \titlepage
\end{frame}

\begin{frame}{Presentation Aims}

	\begin{enumerate}
		\setlength\itemsep{1.25em}
		\item Explore potential mechanisms by which gastrointestinal (GI) events are associated with ibuprofen absorption (make it clear that we're looking at local behavior) (HHSF223201510157C).
		\item Associate these events with GI pH and ibuprofen concentrations with variation in plasma ibuprofen concentrations (HHSF223201510157C and HHSF223201310144C).
		\item Relate these mechanisms to continuing future work on a in vitro GIS product dissolution device (HHSF223201310144C).
	\end{enumerate}

\end{frame}


\begin{frame}\frametitle{Gastrointestinal mechanisms associated with ibuprofen absorption}

	We have identified three ways by which variation in GI activity may associate with variation in plasma concentrations.

	\bigskip

	\begin{enumerate}
		\setlength\itemsep{1em}
		\item \textcolor{blue}{Emptying mechanism}: Increased gastric motility is associated with ibuprofen availability for absorption in the small intestine.
		\item \textcolor{blue}{Intestinal distributional mechanism}: Increased GI motility distributes and spreads ibuprofen along the intestine increasing surface area, which increases plasma absorption rates.
		\item \textcolor{blue}{Dissolution mechanism}: Increased small intestine (SI) motility increases the ibuprofen dissolution rate, which increase the total ibuprofen available for absorption into the plasma.
	\end{enumerate}


\end{frame}

\begin{frame}\frametitle{Model for ibuprofen absorption}

	\small

	In order to understand how these mechanisms related to the ibuprofen absorption, one might consider the path that ibuprofen takes before it reaches the plasma.

	\includegraphics[width=.95\textwidth]{ibuprofen_path.png}

	\pause

	\footnotesize
	\begin{columns}[t]
		\begin{column}{.3\textwidth}
			{\bf We have some understanding of Step 1.}
		\end{column}
		\begin{column}{.3\textwidth}
			{\em Continuing mechanistic analysis addresses this step; Dr. Brasseur's computational simulations concern Step 2.}
		\end{column}
		\begin{column}{.3\textwidth}
			{\bf We have a strong understanding of Step 3}
		\end{column}
	\end{columns}

\end{frame}


\begin{frame}{Motility associations with ibuprofen and absorption}

	Drug distribution can be associated with both global and local measures of GI conditions and motility:

	\bigskip

	\begin{columns}[t]
		\begin{column}{.5\textwidth}
			\textcolor{blue}{Global measures}:
			\footnotesize
			\begin{itemize}
				\item Plasma C-max or C-max/AUC
				\item Plasma T-max
				\item Average jejunal pH
				\item Time to first MMC phase 3 event post dosing.
			\end{itemize}
		\end{column}
		\begin{column}{.5\textwidth}
			\small
			\textcolor{blue}{Local measures}:
			\footnotesize
			\begin{itemize}
				\item Instantaneous absorption rate estimated through deconvolution
				\item Duodenal ibuprofen solution concentration one hour after dosing.
				\item A motility index summarizing jejunal motility for 15 minute period.
			\end{itemize}
		\end{column}
	\end{columns}


	\bigskip

	Ongoing analysis studies the proposed mechanisms using the local measures of intestinal ibuprofen distribution, GI pH, plasma ibuprofen concentrations and motility.

\end{frame}

\begin{frame}\frametitle{Study data: Overview}
	\begin{itemize}
		\item Modeling local measures of drug distribution means that we take advantage of the longitudinal natures of the study data.
		\item Today we will limit our analysis to the fasted state subjects.
		\item Lots of missing data.
		\item summarize number of observations in a variety of plots.
	\end{itemize}

\end{frame}

\begin{frame}{Discrete wavelet transform (DWT) based motility index}

\small

Index aims to capture meaningful variation in manometry data in order to characterize periods of greater and lesser GI activity.

\bigskip

\textcolor{blue}{Method: }
\begin{itemize}
  \item Calculate the DWT representation of a two minute block of manometry data. 
  \item Calculate the motility's energy in the 3 second band (half the period of a typical GI contraction), by squaring and summing the wavelet coefficients.
\end{itemize}

\bigskip

Using these two minute summaries, we can form aggregate motility indices for longer periods of time.



\end{frame}


\begin{frame}{Motility index comparison}

\small

We now compare the DWT based motility index to an alternative index. The alternative motility index is defined as

\begin{center}
$(\#\text{ of contractions}) \times (\text{average pressure of contractions})$.
\end{center}

We calculate the indices on two minute windows of manometry data and compare each through the scatterplot below.
\includegraphics[width=\textwidth]{dwt_comparison.pdf}

\end{frame}

\begin{frame}\frametitle{Methods: Statistical models}

\small
We use \textcolor{blue}{linear mixed effects models} to study trajectories of GI ibuprofen concentrations and ibuprofen plasma absorption rates.

\bigskip

\begin{itemize}
	\item \textcolor{blue}{Mixed effects}: 
	\begin{itemize}
		\item allow associations between dependent and independent variables to vary between subjects.
		\item account for repeated measures within subject.
	\end{itemize}
	\item \textcolor{blue}{Linear:}
	We model proportional relationships, e.g.\\
	\smallskip
	\footnotesize
	Doubling $X$ leads to a 41\% increase in $Y$.
	\small

\end{itemize}

\begin{itemize}
	\item Our analysis today focuses on estimates of population effects, rather than subject specific effects.
	\item Continuing work on in vitro device will likely be interested in inter-subject variation captured by mixed effects models.

\end{itemize}

\end{frame}

\begin{frame}

\LARGE
\begin{center}
\textcolor{blue}{Stage 1: The emptying mechanism}
\end{center}
\end{frame}


\begin{frame}[c]{Emptying Mechanism}

We explore whether gastric motility is associated with subsequent ibuprofen concentrations in the small intestine.

\bigskip

Ibuprofen must reach small intestine for absorption. Variation in gastric emptying rates is first mechanism that can induce variation in ibuprofen exposure.
% \textcolor{blue}{Mechanism:}
% \begin{itemize}
% 	\item [] 
% \end{itemize}

\bigskip

\textcolor{blue}{Study data used investigate the emptying mechanism:}
\begin{itemize}
	\item {\bf Dependent variable}: Total small intestine  ibuprofen concentration at time $t$.
	\item {\bf Independent variable}: Gastric motility during 15 minutes proceeding time $t$. Gastric motility is summarized using the DWT motility index.
\end{itemize}

\end{frame}

\begin{frame}[c]{Emptying mechanism: exploratory data analysis (EDA)}

\includegraphics[width=\textwidth]{mi_vs_conc_by_time.png}

\end{frame}


\begin{frame}[c]{Emptying mechanism: EDA }
\begin{columns}
\begin{column}{.4\textwidth}

We will additionally account for the underlying time trend; On average, after dosing small intestine ibuprofen concentrations increase.

\bigskip

The dashed \textcolor{red}{red} line represents the point-wise average log total small intestine ibuprofen concentration.
\end{column}
\begin{column}{.6\textwidth}
\includegraphics[width=\textwidth]{time_vs_total_conc.png}
\end{column}
\end{columns}
\end{frame}

\begin{frame}{Mechanism one: EDA}
\begin{columns}
\begin{column}{.4\textwidth}
\small
Concentration measurements tend to cluster within individual after accounting for time trends (ICC $\approx$ .4). 

\bigskip

Subjects experience either greater or lower ibuprofen concentrations rather than fluctuating around population average.

\bigskip

Analysis techniques will need to account for correlated nature of data.

\end{column}
\begin{column}{.6\textwidth}
\includegraphics[width=\textwidth]{residual_dot_plot.png}
\end{column}
\end{columns}

\end{frame}


\begin{frame}{Emptying mechanism: results}
\small
\textcolor{blue}{Model:} At time $t$ for subject $j$, $$[\text{log(SI conc.)}]_{t,j} \sim (\beta_1 + \beta_{1,j})[\text{log(time)}]_j + \beta_2[\text{Gastric Mot.}] + \epsilon_{j,t} + \alpha_j,$$

where $\alpha_j \sim N(\beta_0, \tau^2)$, $\beta_j \sim N(0, \sigma_1^2)$, and $\epsilon_{j,t} \sim N(0, \sigma^2)$.% and  $[\alpha_j, \beta_{j}] \sim N\left(\begin{bmatrix} 0 \\ 0 \end{bmatrix}, \begin{bmatrix} \tau^2 & \rho \\ \rho & \sigma_1^2 \end{bmatrix}\right)$.

% latex table generated in R 3.3.2 by xtable 1.8-2 package
% Tue Oct 31 18:24:08 2017
\footnotesize
\begin{table}[ht]
\centering
\begin{tabular}{rrrr||rrr}
  \hline
 & Estimate & Std. Error & Z-Score & $\hat{\sigma}$ & $\hat{\sigma_1}$ & $\hat{\tau}$ \ \\ 
  \hline
Intercept ($\beta_0$) & 9.99 & 0.36 & 27.90 & \multirow{3}{*}{1.54}& \multirow{3}{*}{.85}& \multirow{3}{*}{1.27}\\ 
  log(Time) ($\beta_1$) & 1.60 & 0.29 & 5.59 & & &\\ 
  Gastric Motility ($\beta_2$) & 0.40 & 0.19 & 2.12 & & & \\ 
   \hline
\end{tabular}
\end{table}

\small
\begin{itemize}
	\item Residual ICC = 0.40
	\item Prediction accuracy to multiplicative factor $e^{1.54} = 4.7$.
	\item On average a 1 standard deviation increase of the gastric motility index leads to a 50\% increase in small intestine ibuprofen concentration (SNR = 0.32).
\end{itemize}

\end{frame}

\begin{frame}{Discussion: The emptying mechanism}
\small

% \textcolor{blue}{Interpretations:}
\begin{itemize}
	\setlength\itemsep{1em}
	% \item Given the tablet dosage form, we know that all ibuprofen observed in the small intestine was once in the stomach.
	\item Our analysis shows that greater gastric activity is associated with greater total ibuprofen concentrations in the small intestine, controlling for subject-specific time trends.
	\item GIS device should take into account variability in ibuprofen appearance. There are better modalities (e.g. SmartPill or MRI) for measuring gastric emptying.
\end{itemize}


\end{frame}


\begin{frame}
\begin{center}
\LARGE \textcolor{blue}{Stage 3: Ibuprofen plasma absorption}
\end{center}
\end{frame}


\begin{frame}{Stage 3: Ibuprofen plasma absorption}

\small

The final stage in ibuprofen's path through the GI tract is the absorption into the blood across the mucosa of the small intestine.

\smallskip

Absorption rate variation causes variation in plasma concentrations. Identifying factors associated with absorption rate variation is critical to understand variation in plasma ibuprofen concentrations.

\bigskip

% \begin{enumerate}
% 	\item Gastric emptying moves ibuprofen from stomach to small intestine.
% 	\item SI motility spreads drugs along the small intestine and increases absorption. 
% \end{enumerate}

\textcolor{blue}{Study data used to investigate ibuprofen plasma absorption}
\begin{itemize}
	\item {\bf Dependent variable}: Plasma absorption rate at time $t$. Calculated using the Loo-Riegelman deconvolution.
	\item {\bf Independent variables}: Small intestine
	\begin{itemize}
		\item solution concentration at time $t$.
		\item undissolved concentration at time $t - \delta_1$.
		\item Average pH over $[t - \delta_2, t]$.
		\item motility summarized through the DWT.
	\end{itemize}
\end{itemize}

\end{frame}

\begin{frame}{Ibuprofen plasma absorption: EDA}

\includegraphics[width=\textwidth]{mi_vs_abs_rate_by_time.png}

\end{frame}

\begin{frame}{Ibuprofen plasma absorption: EDA}

\begin{columns}
\begin{column}{.5\textwidth}
\includegraphics[width=\textwidth]{jph_vs_abs_rate.png}
\end{column}
\begin{column}{.5\textwidth}
\includegraphics[width=\textwidth]{dph_vs_abs_rate.png}
\end{column}
\end{columns}



\end{frame}

\begin{frame}{Ibuprofen plasma absorption: EDA}
\small
\begin{columns}
\begin{column}{.5\textwidth}

Log small intestine solution concentration has strongest association with Log plasma absorption rates.

\bigskip

Given the strong causal dependence between drug availability and absorption rates, exploring associations between absorption rates and other factors must be done controlling for the ibuprofen solution concentration.


\end{column}
\begin{column}{.5\textwidth}
\centering
\includegraphics[width=\textwidth]{si_vs_abs_rate.png}
\end{column}
\end{columns}
\end{frame}

% \begin{frame}{Mechanism 2: Modeling approach}
% \small
% We present a series of models that attempt to answer whether we can see mixing/distributional effects of gastric and small intestine motility.

% \bigskip

% We aim to answer the following questions: Do we observe that:
% \begin{enumerate}
% 	\item gastric emptying is associated with absorption rates?
% 	\item small intestine motility is associated with greater absorption rates by distributing ibuprofen along the small intestine?
% 	\item small intestine motility is associated with greater absorption rates by increasing solid ibuprofen dissolution?
% \end{enumerate}

% \end{frame}

\begin{frame}{Associations between gastric emptying and ibuprofen absorption}

\small

We present the results of three statistical models that explore the role of gastric motility/emptying in ibuprofen absorption.

\begin{itemize}
	\item Each row in the table below represents the results of one model fit.
	\item {\bf Standardized} regression coefficients are presented. \textcolor{orange}{Orange} and Black Z-scores denote statistically significant and insignificant associations, respectively. 
\end{itemize}

\footnotesize
\begin{table}[ht]
\centering
\begin{tabular}{r||rrr|r}
Response & Log(time) & Log(SI sol'n conc.) & Gastric Mot & $\hat{\sigma}$\\
\hline
\multirow{3}{*}{Log(Abs. Rate)} & \textcolor{orange}{2.8} & \textcolor{orange}{15.1} & & 1.0\\
 & \textcolor{orange}{4.1} &  & \textcolor{orange}{2.9} & 1.0\\
 & \textcolor{orange}{2.6} & \textcolor{orange}{16.5} & 1.8 & 1.0\\
\hline
\end{tabular}
\end{table}
\small

Controlling for small intestine ibuprofen solution concentration, gastric motility is not significantly associated with ibuprofen absorption rates.

\end{frame}

\begin{frame}{Ibuprofen absorption: modeling results}
\small
\textcolor{blue}{Model:} At time $t$ for subject $j$, $$[\text{log(Abs. rate)}]_{t,j} \sim (\beta_1 + \beta_{1,j})[\text{log(time)}]_j + \beta_2[\text{log(SI sol'n conc.)}] + \epsilon_{j,t} + \alpha_j,$$

where $\alpha_j \sim N(0, \tau^2)$, $\beta_j \sim N(0, \sigma_1^2)$, and $\epsilon_{j,t} \sim N(0, \sigma^2)$.% and  $[\alpha_j, 
\footnotesize
\begin{table}[ht]
\centering
\begin{tabular}{rrrr||rrr}
  \hline
 & Estimate & Std. Error & Z-Score & $\hat{\sigma}$ & $\hat{\sigma_1}$ & $\hat{\tau}$ \ \\ 
  \hline
	log(SI sol'n conc.) & 0.39 & 0.03 & 15.6 & \multirow{2}{*}{1.01}& \multirow{2}{*}{0.56}& \multirow{2}{*}{0.99} \\ 
	log(Time) & 0.68 & 0.25 & 2.8 & & &\\ 
   \hline
\end{tabular}
\end{table}

\small
\begin{itemize}
	\item Residual ICC = 0.49
	\item Prediction accuracy to multiplicative factor $e^{1.01} = 2.75$.
	%\item On average, doubling small intestine ibuprofen solution concentration is associated with a 31\% increase in ibuprofen absorption rate.
	\item Increasing the log(SI solution concentration) by one standard error (3.8 log(ng/ml)) is associated with a 440\% increase in ibuprofen absorption rate (SNR = 1.60).
	\item Bart recommends a summary sentence
\end{itemize}

\end{frame}

\begin{frame}{Discussion of ibuprofen absorption}

\small

	\begin{itemize}
	\setlength\itemsep{1em}
		\item After controlling for contemporaneous small intestine ibuprofen solution concentration, other GI variables are not significantly associated with plasma absorption rates.

		\item This suggests that other variables influence plasma concentration levels through their associations with solution concentrations.

	\end{itemize}
\end{frame}


\begin{frame}{Summary}

\end{frame}

\begin{frame}{Future work}

\end{frame}


\end{document}
